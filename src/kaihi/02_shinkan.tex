\subsection*{申請理由・使用方法}
\writtenBy{\subPresident}{藤原}{浩一}

この度徴収を申請した特別会費は,新歓方針に記載したように二次企画に実施する夕食交流会の予算として使用する.

特別会費として申請した理由として,以下の2点が挙げられる.
\begin{itemize}
    \item 会員への還元性
    \item 一部の会員に支払いを負担させる可能性
\end{itemize}

\begin{description}
    \item[会員への還元性]\mbox{}\\ %文章のスタイルの関係でエラーを無視
        通常の会費は本会に所属する全ての会員に対し,等しく還元されるものに対して執行される必要がある.
        しかしながら,夕食交流会に関しては本会主催ではあるが,参加対象者は未入会の新\firstGrade{}であるため,
        会員に還元されている訳でない.

    \item[一部の会員に支払いを負担させる可能性]\mbox{}\\ %文章のスタイルの関係でエラーを無視
        夕食交流会は,新\firstGrade{}に本会の魅力を感じ,上回生と交流を深めてもらうことを目的としており,
        新\firstGrade{}の参加費を500円としている.
        しかしながら,全ての上回生が参加しているわけでないため,事前に一定の金額を徴収していないと,
        一部の会員に支払いを負担させてしまう可能性がある.

 \end{description}

上記の理由から,今回は使用用途を限定した特別会費を徴収することを申請した.
なお,徴収した特別会費は原則新\firstGrade{}の食事代の補助に充てるものとし,
上回生の参加費は自己負担とする.

また,新\firstGrade{}の参加人数によっては,余剰が発生する可能性がある.
その場合に関しては,通常の会費と同様に取り扱うものとする.
