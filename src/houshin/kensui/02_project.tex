\subsection*{プロジェクト方針}

%\writtenBy{\kensuiChief}{中尾}{龍矢}

本項では本局におけるプロジェクト活動業務に関する2020年度春学期の方針を以下の点において述べる.

\begin{itemize}
\item 企画書の募集
\item 週報の回収・催促
\item 会員のプロジェクト管理
\item 発表の機会の提供
\item 報告書の回収
\end{itemize}

\subsubsection*{企画書の募集}

問題ないため例年通りに実施する。\\
\subsubsection*{週報の回収・催促}
週報の回収・催促はSlackを通じて行い,場合によって催促は定例会議でも行う.\\全体として提出率が良くなく、催促のし忘れも見られたため、幾つかの強化を実施する.\\具体的な方法としては、Slackのダイレクトメッセージ、または直接催促を行う.\\催促のし忘れについては、局会議にて確認を行う.\\


\subsubsection*{会員のプロジェクト管理}

プロジェクトの異動および,途中参加には本局と異動先のプロジェクトのリーダーの承認が必要である.\\また,プロジェクトのリーダーの負担を減らすため,要望があれば本局員が活動を行う部屋の予約を代行する.\\週報を通して,活動が芳しくないプロジェクトがあれば本局の方から上回生会議で報告し,上回生会議にて適切な処置を図る.\\今回はDTM班のプロジェクトリーダーの失踪のため、週報の回収や報告書の作成名などその他の作業に支障をきたしたため、リーダー失踪に対する対策が検討された.\\具体的な対策方法は、リーダー失踪を事前に察知を目的とする情報共有を行う.\\

\subsubsection*{発表の機会の提供}

今年度前期と同様に追い込み合宿とプロジェクト発表会を行う.プロジェクト発表会の発表順については定例会議やSlackを用いて通達する.\\発表時間は10分以上30分未満とし,発表会全体の進行に支障をきたすようであれば該当する発表を中止または終了を催促することができるものとする.\\追い込み合宿直近の定例会議において,研究推進局は報告書の書き方及びプロジェクト発表会に必要なものに対して再通知を行うものとする.

\subsubsection*{報告書の回収}

プロジェクトは報告書の提出をもって完了とする.\\報告書の提出はPDFで行うものとし,報告書の執筆については,PDF形式で書き出せるものであれば,制限を設けないものとする.\\報告書に関しては,研究推進局から各プロジェクトに対し,報告書作成のためのテンプレートを用意するものとする.\\報告書の回収はプロジェクト発表会までに行うものとする.また,各班は報告書を2部印刷し,プロジェクト発表会に持参することを義務とする.\\

%\writtenBy{\kensuiStaff}{中尾}{龍矢}
