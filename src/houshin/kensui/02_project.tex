\subsection*{プロジェクト方針}

\writtenBy{\kensuiStaff}{中尾}{龍矢}

本項では本局におけるプロジェクト活動業務に関する2020年度春学期の方針を以下の点において述べる.

\begin{itemize}
\item 企画書の募集
\item 週報の回収・催促
\item 会員のプロジェクト管理
\end{itemize}

\subsubsection*{企画書の募集}
問題ないため例年通りに実施する.

\subsubsection*{週報の回収・催促}

週報の回収・催促はSlackを通じて行い,場合によって催促は定例会議でも行う.
全体として提出率が良くなく,催促のし忘れも見られたため,提出を催促する体制を強化するために、t気のような対策を行う.
具体的な方法としては,Slackのダイレクトメッセージ,または直接催促を行う.
催促のし忘れについては,局会議にて確認を行う.

\subsubsection*{会員のプロジェクト管理}

プロジェクトの異動および,途中参加には本局と異動先のプロジェクトのリーダーの承認が必要である.
また,プロジェクトのリーダーの負担を減らすため,要望があれば本局員が活動を行う部屋の予約を代行する.
週報を通して,活動が芳しくないプロジェクトがあれば本局の方から上回生会議で報告し,上回生会議にて適切な処置を図る.
2019年度秋学期はDTM班のプロジェクトリーダーが失踪し,週報の回収や報告書の作成名などその他の作業に支障をきたしたため,リーダー失踪に対する対策が検討された.
具体的な対策方法は,リーダー失踪を事前に察知すべく、情報共有を行っていく.
