\subsection*{全体方針}
\writtenBy{\kensuiStaff}{八木田}{裕伍}

2020年度春学期の活動の中心になるものとして以下の三つを挙げる.
\begin{itemize}
\item 平常活動の支援
\item 会員が興味関心のある活動ができる環境づくり
\item 発信力を養うための環境づくり
\end{itemize}

\subsubsection*{平常活動の支援}
平常活動の支援に関しては,週報を用いてプロジェクト活動の進捗確認や問題の有無の確認を行う.
問題が確認された場合は,それを上回生会議の議題に上げることで問題の解決を図る.
2019年度秋学期では一部の班に週報の提出の遅れや未提出があったため,
2020年度春学期ではプロジェクトリーダー個人に対してのリマインドを強化することで解決を図る.
また,2019年度秋学期では局員がリマインドを失念する事態があったが,リマインドを実行するタイミングを局会議内にすることにより再発を防止する.

追い込み合宿において,エポック立命21の交流室や各宿泊部屋の開放時間や開け方,生協施設などの利用時間,
使用可能な部屋と男女それぞれの部屋割りについて会員がわからないといった問題が発生した.
2020年度春学期においては,これらの情報についての周知を徹底することで対処する.

\subsubsection*{会員が興味関心のある活動ができる環境づくり}
会員が興味関心のある活動ができる環境づくりに関しては,開催を希望する勉強会のアンケートをとり,
多くの意見が寄せられた分野の環境を充実させることで実現していく.

\subsubsection*{発信力を養うための環境づくり}
発信力を養うための環境づくりに関しては,2019年度秋学期と同様にLTを実施する.
会員の意欲向上のために,会員それぞれに質が高いと感じたLTや,
面白いと感じたLTを一つ投票してもらう.
そして一番票が多かった会員に対して,本会で購入する本の選択権を与える.
