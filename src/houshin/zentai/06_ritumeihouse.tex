\subsection*{立命の家方針}

\writtenBy{\firstGrade}{八木田}{裕伍}

\subsubsection*{概要}
立命の家は毎年立命館大学で開催される小学生を対象とした
企画である.2019年は8月23日,24日の二日間開催された.
立命館大学に所属する学術団体やプロジェクト団体は
小学生に科学や英語,ものづくりの楽しさを知ってもらうために
様々な企画を行い,小学生と交流を深める.

\subsubsection*{目的と目標}
立命の家企画に参加する目的は,本会が学術部公認団体として
求められる還元活動の義務を果たすことである.
この目的の達成のために,本企画に参加する小学生に
本会の活動と情報技術に興味関心を向けてもらえることを目標とする.
加えて,小学生に教える体験を通して本会員の教える能力の向上をはかり,
今後の還元活動を円滑にできるようになることを目標とする.

\subsubsection*{実施内容と提案}
2019年は「Scratch」を用いて,小学生にプログラミングを体験してもらった.
以前は,「Code.org」「プログラミン」「Scratch」の三つのサービスを順番に利用する方針となっていたが,
「Code.org」については会内における詳しい者の不在,
「プログラミン」についてはサービス自体の終了といった理由により,
3年連続「Scratch」を使用している.
しかし,3年連続同じサービスであることにより生じる問題点として,
例年参加している小学生へ提供する新規性が少なくなってしまうことが挙げられる
これを解決するために,
今年は5,6年生といった高学年向けに,
「Scratch」に加えて,
ビジュアルプログラミング言語以外のプログラミング言語を使用することとする.
具体的には,JavaScriptとそのクリエイティブコーディングライブラリである「p5.js」を使用する予定である.
選定理由としては,以下の4点を挙げる.

\begin {itemize}
    \item 使用する文法が比較的単純である
    \item コーディングによる成果がビジュアル的にわかりやすい
    \item オンラインコーディング環境があり,環境構築がしやすい
    \item 現在汎く使用されているJavaScriptを学べることの小学生及び保護者への訴求力が高い
\end {itemize}

\subsubsection*{改善点}

2020年からの改善点として,以下の2点を挙げる.

\begin {itemize}
    \item 教材など事前準備の不足
    \item 企画担当者及び企画スタッフの人員不足
\end {itemize}

人員不足に関しては,ある程度の開催時期を把握しておき,
早いうちから引率スタッフと企画スタッフの募集をかける.
また,例年,立命の家担当者は実行委員会の仕事で毎年本会の企画に参加できないため,
2020年は2019年に引き続き別途企画担当者を決める.

事前準備の不足については,早期に小学生向けに,Scratch及びp5.jsの説明や問題の資料作成を行う.
企画スタッフも当日小学生に教えることが出来るよう,事前にScratch及びp5.jsの勉強会などを行う予定である.

\subsubsection*{役割}

ここでは,その概要を仕事別に述べる.

\begin{itemize}
 \item 担当者
  \begin{itemize}
  \item 立命の家担当者は立命の家実行委員として参加する.週1回の実行委員会に出席し,立命の家全体の企画と運営をする.
  \end{itemize}
 \item 引率スタッフ
 \begin{itemize}
  \item 引率スタッフは実行委員会主催の事前説明会に参加し,立命の家における動きや注意事項についての説明を受ける.
  \item 当日は各サークルの企画会場へ引率し,小学生との交流を行う.
  \end{itemize}
 \item 企画リーダー
  \begin{itemize}
  \item 企画を考え,主導する.
  \item 当日は企画の進行をする.
  \end{itemize}
 \item 企画スタッフ
  \begin{itemize}
  \item 企画リーダーのサポートをする.
  \item 当日は小学生たちに企画の説明をし,小学生からの質問があれば対応する.
  \end{itemize}
\end{itemize}
