\subsection*{運営方針}

\writtenBy{\secondGrade}{西見}{元希}

2020年度の運営について,以下の4点から方針を述べる.
\begin{itemize}
    \item 定例会議
    \item 上回生会議
    \item 局
    \item 企画
\end{itemize}

\subsubsection*{定例会議}
2020年度においても,2019年度同様週一回の定例会議を行う.

2019年度同様Slack内の専用チャンネルに自由に議題投稿を行いそれに沿った形での議事進行に努め,局や企画からの連絡や会員全体ですべき議決,LTなどを行う.
また,2019年度秋学期に試験的に導入されていた回答フォーム確認サービスの活用にも力を入れていく.


\subsubsection*{上回生会議}
2020年度においても,週一回の頻度にて上回生会議を行う.
執行部および提出された議題関係者は全員参加とする.
欠席の場合は必ず代理人を立てるようにする.
議決権のない会員に関しても2019年度同様参加の意志があればその出席を認めることとする.
会議で取り扱う企画書の提出は2020年度春学期は前日までと定め,企画書のレビュー時間の短縮と円滑な議事進行に努める.

\subsubsection*{局}
2020年度も2019年度同様春学期末の早期局配属を実施する.
目的は局会議や業務を通じ新入生との親睦を深めることと各局の業務引き継ぎを円滑にするためである.
6月末までに局配属希望調査を実施し,その後面談を行った上で春学期末までに配属を行う.
新入生の局活動への参加タイミングは各局に委ね,夏期長期休暇期間にも引き継ぎ作業を行えるようにする.
局長は局会議を実施し,上回生会議内での議題を共有し局員が同局に関する議題内容を把握できるようにする.


\subsubsection*{企画}
本会外部との関わりがある行事に関しては\secondGrade{}の中で担当者を定め,その企画を行う.
各企画について二人以上担当者をおくようにし,2019年度の同企画担当者がそのサポートを行う.
企画書提出から企画終了までは進捗確認のため担当者は上回生会議にてその度合いを報告する.
また,企画終了後には振り返りとしてKPTを行い,引き継ぎ資料の作成に努める.
