\subsection*{学園祭方針}


\writtenBy{\firstGrade}{齋藤}{竜也}
\subsection*{目的}
\begin{itemize}
    \item 学術部公認団体としての還元活動
    \item 本会の能力向上
    \item サークル外部への情報発信
    \end{itemize}
これらを目的に学園祭を運営していく.この目的を達成するために以下の事を目標にする.
\subsection*{目標}
\begin{itemize}
    \item 来場者数150名
    \item 会誌頒布数100部
    \item アンケート回収率7割
    \end{itemize}
\subsection*{企画}
学園祭の企画として以下のものを行う.
\begin{itemize}
    \item プロジェクト活動発表
    \item LT
    \item 制作物の展示
    \item 開始の頒布
    \item アンケートの回収
    \end{itemize}
\subsubsection*{プロジェクト活動発表}
2019年度と同様に2020年度でもスライド・ポスターを用いたプロジェクト活動発表を行う.学園祭では
一般向けのスライドを作成する事を意識し行う.発表を通して,プロジェクト活動の理解をより深める・発表機会を作る
などの目的の元,原則\firstGrade{}が行う.また,適した発表であるかレビューする時間を設ける.

\subsubsection*{LT}
2019年度と同様に希望者を募り,制限時間5分のLTを行う.原則LTのテーマは情報系に限る.
また聞き手が分かりやすいように考慮したLTを心掛ける.

\subsubsection*{制作物の展示}
2019年度では,制作物の展示数が乏しい結果となっていた.2020年度では,\secondGrade{}
に制作物の提出を求める.他の回生については任意提出とする.

\subsubsection*{会誌の頒布}
2019年度と同様に2020年度でも来場者に対して会誌の頒布を行う.

\subsubsection*{アンケートの回収}
KPTや2021年度以降にも反省を活かす目的でアンケートを回収する.
回収率をより高めるため,会誌の頒布と併せて行うなどの工夫をする.

\subsection*{広報物}
2019年度では展示する制作物が決定しておらず,広報物に反映できなかったため,早期から
制作物の催促を行う.また,広報物に関しては担当の班を割り振り,流暢な広報物作成を行う.
\begin{itemize}
    \item ビラ
    \item ポスター
    \item 人間広告
    \item 動画
    \item 本会Webサイト
    \item 本会Twitter
    \end{itemize}
\subsubsection*{ビラ}
印刷は100枚を目安に作成する.

\subsubsection*{ポスター}
張り出しは学園祭の事前に行う.終了後は早急に回収する.
2019年度ではポスターの内容に困惑している班員も見受けられた.
場所・日時・企画内容・制作物などについて記載する.
2019年度はポスターサイズの印刷が間に合わなかったため,こまめに呼び掛けを行う.

\subsubsection*{人間広告}
看板を作成し,看板を用いて広告する.2019年度と同様の形式で行う.

\subsubsection*{動画}
2020年度も2019年度同様の形式で動画作成を行う.動画はWebサイト・会場で公開する.

\subsubsection*{本会Webサイト}
主に渉外局の担当である.本会のトップページを学園祭に置き換える・
動画の公開などを行う.

\subsubsection*{本会Twitter}
Webサイトと同様,主に渉外局の担当である.
Twitterでは開催しているLTやプロジェクト発表など,より細かなイベント告知を行う.

\subsubsection*{レイアウト}
レイアウトに関しては例年,使用する部屋が決定する前にレイアウト提出を求められるため,
2019年度のレイアウトを参考に提出する.部屋の決定後必要に応じてレイアウトを変更する.

