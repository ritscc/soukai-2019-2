\subsection*{Welcomeゼミ方針}

%\writtenBy{\president}{阿部}{竜也}
%\writtenBy{\subPresident}{阿部}{竜也}
\writtenBy{\firstGrade}{阿部}{竜也}
%\writtenBy{\secondGrade}{阿部}{竜也}
%\writtenBy{\thirdGrade}{阿部}{竜也}
%\writtenBy{\fourthGrade}{阿部}{竜也}

Welcomeゼミを行う目的は,以下の2点である.

\begin{itemize}
    \item 新入生に本会に興味を持ってもらうこと
    \item 新入生に本会の活動を体験してもらうこと
\end{itemize}
これは新入生に本会の方針を知ってもらうとともに,
本会の活動を体験してもらった上で入会してもらうためにも重要である.

\subsubsection*{目標}
目標に関しては,
\begin{itemize}
    \item 気軽にサークルルームに来てもらうこと
    \item 新入生にとってサークルルームが居心地がいい空間にすること
    \item 新入生の中長期的な定着
\end{itemize}
の三点を挙げ,
新入生に本会に馴染んでもらえるようにしたい.

\subsubsection*{手法}
この項では目標を達成するために,具体的な手法について,形式,内容に分けて述べる.

 まず,形式については,例年通り新入生の希望分野に適した上回生をあてがい開発を行う.
担当する新入生の力量によって上回生の負担が左右されるため,1人に負担が集中することのないよう,
新入生の進捗を上回生全体で共有,管理を行う.

 次に,内容については,新入生にやりたいことをアンケート形式で募ることによって上回生の割り振りを考え,
可能な限り希望通りの分野の体験ができるようにする.
新入生がより円滑に進捗を進めるために,上回生間で新入生の進捗の共有を図る.
担当の上回生は,学内メールなどによって新入生と積極的に連絡を取ることによって新入生の進捗の把握を行い,上回生間に共有する.

 また,Welcomeゼミ期間終了後の最初の定例会議において原則全員参加の成果物発表会を行い,
新入生に技術の共有の場の体験をしてもらうこととする.

