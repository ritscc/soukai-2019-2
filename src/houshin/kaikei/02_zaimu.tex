\subsection*{財務方針}

\subsubsection*{局会議}
週に1度会計局員が集まる日を設け,会議を行う.

\subsubsection*{会計情報の公開}
定例会議および上回生会議中に予算と執行額,予算執行率をGoogleドライブ上で公開する.
より詳細な会計情報は会計局員を通して公開する.
また,新入生の本会の予算目的での入会を防ぐ為に5月末までは上回生会議でのみ公開する.

\subsubsection*{会計ガイダンス}
2020年度春学期定例会議中に会員に対して一回行う.必要に応じて2020年度秋学期にもガイダンスを行う.
会計ガイダンスは領収書の切り方,購入申請の方法,3Dプリンターとプリンターについての説明,会費の使用用途,学友会費について知ってもらうことを目的として行う.

\subsubsection*{引き継ぎ文書}
2019年度までに作成された引き継ぎ文書に必要に応じて加筆・修正を加える.

\subsubsection*{会費}
新規入部者は,入会フォーム(Googleフォーム)と会費徴収を同時に行った後に入会届の書類を発行し,署名してもらう.
その後,執行部員が一時受理し,書類を保存する.
最後に,執行委員長に受理されれば入部手続きが完了することとする.
本会に参加したことのある会員は入会フォーム(Googleフォーム)と会費を徴収することとする.
会員から年額¥6,000を会費として入会届と同時に徴収する.
集金システムは2019年度のものを引き続き利用する.

\subsubsection*{購入申請}
購入申請とは会員が購入したい商品を本会予算から購入できる制度である.
申請方法としては本局員に申請を行い,本局で審議を行う.
その後定例会議でプレゼンテーションを行ってもらい決議を採る.
また,本会の活動に明らかに必要かつ低額な物については本局の承認があれば購入できるものとする.

