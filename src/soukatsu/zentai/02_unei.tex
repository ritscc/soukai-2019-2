\subsection*{運営総括}
\writtenBy{\subPresident}{藤原}{浩一}
2019年度秋学期の運営に関して,以下の3点から総括を述べる.
\begin{itemize}
    \item 定例会議
    \item 上回生会議
    \item 企画
\end{itemize}

\subsubsection*{定例会議}
2019年度秋学期に関しても,春学期と同様に,開講期の毎週木曜日の放課後に定例会議を開催した.
開催曜日および時間帯については,欠席人数の少なさを鑑みて問題はなかったと考える.

議題に関連する資料に関しても,事前に上回生会議内で作成を徹底することにより,開催時には間に合っていた.

また,会内では会員が回答すべきGoogleフォームを一覧で表示し,会員ごとに回答済みであるか確認できるサービス「RCCフォームビューア」の提供を開始した.
Slackと連携し,回答の締切日時を通知してくれるため,以前と比較し,締切を守った回答の数が増え,改善されたと考える.

\subsubsection*{上回生会議}
2019年度秋学期の上回生会議は,毎週火曜日の放課後に問題なく開催された.
なお,開催曜日と時間帯に関しては,主な出席者である執行部の都合を擦り合わせ設定した.

取り扱った議題として,3Dプリンタの使用ルールの見直しや,備品管理の担当の見直しなど,
比較的大きな問題に対しても議論を行うことができた.
また,各局でも議題の共有が局会議内で行われ,局員が議題内容を把握できていた.

春学期で問題となった無断欠席については,一部改善がみられたが,未だ問題といえる.
局長だけではないが,上回生会議は本会の運営に関して話し合う重要な会議であるという認識を改めて持つ必要があると考える.
%代理者も必要であるが,そもそも局長が意識してほしい.代理はあくまでも局長の仕事を委任していることを忘れなきよう.

企画書の提出に関しては,春学期にあった締切制度を廃止したが,廃止したことによる問題は発生しなかった.
しかしながら,内容が詰められていないものなど,査読するのに時間を要するものもあったためえ,局会議などで事前の査読を行う必要があると考えている.

\subsubsection*{企画}
企画の進捗確認に関しては,春学期同様,上回生会議にて担当者からの報告を随時受けていたため,執行部内で共有されていた.

また,企画終了後にはKPTを用いた振り返りを行うことができたため,申し分ないと考える.