\subsection*{2019年度後期プロジェクト活動総括}

%\writtenBy{\president}{服部}{瑠斗}
%\writtenBy{\subPresident}{服部}{瑠斗}
%\writtenBy{\firstGrade}{服部}{瑠斗}
\writtenBy{\secondGrade}{服部}{瑠斗}
%\writtenBy{\thirdGrade}{服部}{瑠斗}
%\writtenBy{\fourthGrade}{服部}{瑠斗}

\subsubsection*{全体総括}
2019年度秋学期のプロジェクト活動は,9月中旬頃から企画書の募集を開始し,10月中旬に活動を開始した.
各プロジェクトには活動ごとに週報を提出することを義務付け,進捗確認を行った.
2月10日には各プロジェクトで得た知見や技術を会内で共有するべく,プロジェクト発表会を行った.
全6個のプロジェクト全てが報告書を提出した.
以下に2019年度秋学期に活動していたプロジェクトの一覧を示す.

\begin{itemize}
\item 自作キーボード班
\item メタプログラミング班
\item DTM班
\item ビジュアルノベル班
\item piet challenge班
\item Web班
\end{itemize}

プロジェクト活動の総括は以下の六つに分けて行う.

\begin{itemize}
\item 目標の総括
\item プロジェクトの内容
\item 週報
\item 報告書
\item 追い込み合宿
\item プロジェクト発表会
\end{itemize}

\subsubsection*{目標の総括}
2019年度秋学期の目標は以下の三つであった.

\begin{itemize}
\item 活動を通して技術力の向上を図る
\item グループの一員であることを自覚して活動することで集団行動の重要性を学ぶ
\item 活動によって得られた成果を本会Webサイトを通して公開する
\end{itemize}

これらを踏まえた総括を以下に記す.

活動を通して技術力の向上を図るに関しては,\firstGrade{},上回生共に
班の活動を通して新たな知見を得られた会員が多かったことから,概ね達成できたと言える.
しかし,個人活動になりがちな班や参加者が少ない班が見受けられた.

集団行動の重要性を学ぶに関しては,一部の班において活動の頻度が減少したものの,ある程度は達成はできたと言える.
しかし,班によっては活動報告書の執筆が一部のメンバーに集中するなど,集団で協力できていない点も見受けられた.

得られた成果を本会Webサイトを通して公開するに関しては,予め決めていたスケジュールに則って報告書のレビューなどを実施している.

\subsubsection*{プロジェクトの内容}
プロジェクトの内容については全ての班において適切であった.
上回生会議によって,不適切と思われるプロジェクトは排除したため,設立した班の中で不適切なテーマは無かった.

\subsubsection*{週報}
週報の提出率は後半において提出率が極度に減少した.
これは,研究推進局とプロジェクトリーダーとの連絡を十分に行うことが出来なかった点が要因であると考えられる.
週報に問題点を記載した班は無かったが,一部の班がプロジェクト活動の継続に問題が発生した.
よって週報のみからプロジェクト活動の問題点を抽出することは厳しいと考えられる.

\subsubsection*{報告書}
報告書の提出をもってプロジェクト終了とする.
報告書の必須項目を以下に示す.

\begin{itemize}
\item 活動動機,目的
\item 活動内容
\item 活動結果
\item 考察
\item 参考文献
\end{itemize} 

報告書の内容に関しては,ほぼ問題はなかった.
これは,2019年度春学期での方針を踏まえて研究推進局が報告書の形式の連絡を明確に行ったためであると考えられる.
しかし,班員数に対して執筆者数の少なさが目立つ班が見受けられた.
報告書の形式については,PDF形式で提出するように指定したので,問題は起こらなかった.

\subsubsection*{追い込み合宿}
2月4日,5日にプロジェクト発表会に向けて班員に準備を行ってもらうために,追い込み合宿を行った.
原則全員参加であるが,準備が完了している班に関しては参加しなくても良いものとした.

当日に関しては,予め予定していた時間に交流室を開放することが出来ていなかった.
これは研究推進局長が解放時間を勘違いしていたことが原因だと考えられる.

参加者は2019年度春学期よりも多く,追い込み合宿に参加した会員の大半は,追い込み合宿で進捗を出していたことから
追い込み合宿は十分な有用性があると言える.

\subsubsection*{プロジェクト発表会}
プロジェクト活動を通して得ることができた知見や技術を会内で共有する場としてプロジェクト発表会を行った.

原則全員参加で,参加率はあまり良くなかった.
これは研究推進局からのリマインドの徹底や重要性の連絡が不足していたことが挙げられる.

プロジェクト発表会は報告書を読む時間をとった後に発表,といった形式で行った.
本来ならば質疑応答の時間を設けるのだが,司会側の不手際により実施することが出来なかった.

PCの使用は全員に許可したが,不都合は見当たらなかった.
報告書を読む時間を利用して報告書の校閲も行ったため,より効率的な発表が行えた.
校閲にはGoogleドライブに提出されたPDF上で行った.
そのため,円滑な校閲を行う事ができた.

発表に関しては,半分以上の班が予め発表スライドを作成していたため,円滑に進行した.
しかし,一部の班において発表スライドの内容が濃く,予め決めていた発表時間よりも多くの時間を使って発表した.
これは司会側が発表時間の制限について失念していたことが原因だと考えられる.

