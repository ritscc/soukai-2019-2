\subsection*{2019年度秋学期活動総括}

\writtenBy{\president}{青木}{雅典}

本会の目的である「情報科学の研究,及びその成果の発表を活動の基本に会員相互の
親睦を図り,学術文化の創造と発展に寄与する」を踏まえ,方針として以下の五つを立てた.
これらについてそれぞれ評価を行うことで2019年度秋学期活動の総括とする.

\begin{itemize}
    \item 親睦を深める
    \item 規律ある行動
    \item 自己発信力の向上
    \item 会員間の技術向上
    \item 外部への情報の発信
\end{itemize}

\subsubsection*{親睦を深める}
2019年度秋学期活動では,プロジェクト活動や冬季ハッカソン,クリスマス会,回生別・
局別コンパを通して会員間の親睦を図った.

プロジェクト活動では,一部活動状況の芳しくない班で連携が取れていない状況が
発生したが,多くの班では班員同士で積極的にコミュニケーションを取れていた.
冬季ハッカソンやクリスマス会では,班分けやイベント構成の工夫から,回生の枠に
囚われず縦の繋がりが生まれる良い機会となった.

回生別コンパについては,本来は2019年度春学期に開催する予定であったが,
スケジュールの都合から2019年度秋学期の開催となった.
1〜3回生それぞれで開催し,どの回生でも親睦を深める良い機会となった.
また,局別コンパでは新たに配属された1回生と上回生の間で親睦を深めたほか,
2020年度にむけての人員方針を決める機会となった.

全体として,各イベントごとに会員それぞれが親睦を深める機会を複数設けることが
でき,会全体としての運営力向上に繋がったと考えられる.

\subsubsection*{規律ある行動}
サークルが適切かつ円滑に運営されるために必要な,会員が最低限行うべき行動方針として,
2019年度秋学期活動では遅刻・欠席連絡と備品整備について方針を定めた.

遅刻・欠席連絡については,止むを得ず遅刻あるいは欠席する場合,その理由と
到着予想時刻を明記するよう定めた.2018年度に比べて連絡の多くが開始時刻前に行われ,
少しずつ改善の傾向が見られた.しかし,ほとんどの遅刻連絡で到着予想時刻が明記されず,
また理由も多くが曖昧であった.連絡のない遅刻欠席は減少傾向にあるものの,未だ改善の
余地は多く見受けられる.

また,備品整備については総務局を中心に行い,私物管理の徹底を行った.その他,
定例会議ではサークルルーム内での音出しに関する簡易な注意を行ったが,
勧告後は目立った問題も発生せず,全体として規律は守られていたと言える.

\subsubsection*{自己発信力の向上}
自分の意見を相手に対してわかりやすく伝える能力を身につけることを目的として,
LTや夏期成果物発表会,学園祭,会誌,Advent Calendarを通して自己発信力の向上
を図った.

定例会議におけるLTでは,様々な分野について会員が発表し,また発表されたコンテンツに
対して会員から様々な指摘が行われた.これにより,会員それぞれがLTを通して自身の考えを他者に伝える機会を得た.
夏期成果物発表会では,複数の会員が発表を行ったが,参加者が比較的
少なく十分なフィードバックを得られなかった.学園祭では,LTやプロジェクト発表を
行ったが,集客状況から会内発表と差異の認められない体験であった.会誌については,
執筆に向けてコンテンツを準備した会員も存在し,提出率の面から情報発信や技術力向上の
機会として十分に働いた.Advent Calendarでは,自身の扱った技術的事項について,
インターネット上に公開されることを意識しながら執筆されていた.提出状況については,
その都度渉外局によって管理されており,全体として不足することなく公開されていた.

全体として,各企画において自身の扱う技術的コンテンツを発表する機会が設けられており,
またそれらの企画によって会員が積極的に情報発信を行っていたと言える.

\subsubsection*{会員間の技術向上}
会全体の技術力を向上させることを目的に,プロジェクト活動のほか,夏期成果物発表会,
勉強会,冬季ハッカソンなど,会全体として技術共有の場を設けた.

プロジェクト活動では,それぞれの班で扱った分野について,通常活動や追い込み合宿において
技術力向上の機会が与えられた.夏期成果物発表会では,協賛企業からの技術的発表を
通して技術習得のきっかけを得た.勉強会については,システム管理局が主催するサーバ構築
勉強会が開催された.なお開催希望日が会員から複数挙げられたため,同一内容について
2日開催とした.冬季ハッカソンでは,参加した会員がそれぞれの班で情報技術に関する
知見を共有しながら開発を行う機会となった.

全体として,それぞれの企画において参加した会員が技術力向上の機会を得ていたと言える.

\subsubsection*{外部への情報の発信}
本会の活動について会外へ発信し,活動を知ってもらうことを目的として,
会公式Twitterや本会Webサイトを活用して情報の発信を行った.

会公式Twitterでは,LTやイベントについて写真を交えて積極的に本会の活動を発信していた.
本会Webサイトについては,ロゴとともにリニューアルを行い,刷新したトップページから
各コンテンツを効果的に活用できた.さらに,学園祭や会誌など,会外の方が目にする
コンテンツについては,わかりやすく楽しめるような伝え方とするため,レビューを重ねながら
自由な内容を発信していた.

その他,個人や会員同士が積極的に外部イベントに参加し,本会やそれに関する活動および
技術について発信しており,会全体として外部への情報発信が活発に行われていたと言える.
