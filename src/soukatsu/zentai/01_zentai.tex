\subsection*{2019年度秋学期活動総括}

\writtenBy{\president}{青木}{雅典}

本会の目的である「情報科学の研究,及びその成果の発表を活動の基本に会員相互の
親睦を図り,学術文化の創造と発展に寄与する」を踏まえ,方針として以下の五つを立てた.
これらについてそれぞれ評価を行うことで2019年度秋学期活動の総括とする.

\begin{itemize}
    \item 親睦を深める
    \item 規律ある行動
    \item 自己発信力の向上
    \item 会員間の技術向上
    \item 外部への情報の発信
\end{itemize}

\subsubsection*{親睦を深める}
2019年度秋学期活動では,プロジェクト活動や冬ハッカソン,クリスマス会,回生別・
局別コンパを通して会員間の親睦を図った.

プロジェクト活動では,一部活動状況の芳しくない班で連携が取れていない状況が
発生したが,多くの班では班員同士で積極的にコミュニケーションを取れていた.
冬ハッカソンやクリスマス回では,班分けやイベント構成の工夫から,回生の枠に
囚われず縦の繋がりが生まれる良い機会となった.

回生別コンパについては,本来は2019年度春学期に開催する予定であったが,
スケジュールの都合から2019年度秋学期の開催となった.
1〜3回生それぞれで開催し,どの回生でも親睦を深める良い機会となった.
また,局別コンパでは新たに配属された1回生と上回生の間で親睦を深めたほか,
2020年度にむけての人員方針を決める機会となった.

全体として,各イベントごとに会員それぞれが親睦を深める機会を複数設けることが
でき,会全体としての運営力向上に繋がったと考えられる.

\subsubsection*{規律ある行動}

サークルが適切かつ円滑に運営されるために必要な,会員が最低限行うべき行動方針として,
2019年度秋学期活動では遅刻・欠席連絡と備品整備について方針を定めた.

遅刻・欠席連絡については,止むを得ず遅刻あるいは欠席する場合,その理由と
到着予想時刻を明記するよう定めた.2018年度に比べて連絡の多くが開始時刻前に行われ,
少しずつ改善の傾向が見られた.しかし,ほとんどの遅刻連絡で到着予想時刻が明記されず,
また理由も多くが曖昧であった.連絡のない遅刻欠席は減少傾向にあるものの,未だ改善の
余地は多く見受けられる.

また,備品整備については総務局を中心に行い,私物管理の徹底を行なった.その他,
定例会議ではサークルルーム内での音出しに関する簡易な注意を行なったが,
勧告後は目立った問題も発生せず,全体として規律は守られていたといえる.

\subsubsection*{自己発信力の向上}

\subsubsection*{会員間の技術向上}

\subsubsection*{外部への情報の発信}
