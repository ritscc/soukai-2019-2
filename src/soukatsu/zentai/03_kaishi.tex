\subsection*{会誌総括}

%\writtenBy{\president}{西見}{元希}
%\writtenBy{\subPresident}{西見}{元希}
%\writtenBy{\firstGrade}{西見}{元希}
\writtenBy{\secondGrade}{西見}{元希}
%\writtenBy{\thirdGrade}{西見}{元希}
%\writtenBy{\fourthGrade}{西見}{元希}

本会の活動及びその成果を会外へ発信することを目的として,2019年度も会誌を発行した.
2018年度の学園祭において,来場者の殆どに会誌を頒布することができたため,2019年度は目標頒布数を2018年度より30部多い100部としていた.
2019年度の学園祭においては来場者の約半数に会誌を受け取ってもらうことができ,頒布数は目標の7割である70部に達した.
会誌の内容には,以下の3点を載せた.
\begin{itemize}
  \item プロジェクト活動報告書
  \item 会員によるコラム
  \item 本会の大まかな説明・紹介
\end{itemize}

2018年度同様,会員によるコラム,及びプロジェクト報告書のレイアウトを統一するため,事前にGoogleスライドを用いてテンプレートを作成し会員が執筆しやすい環境を整えた.2018年度の改善点の一つであったブラウザ環境の差異によるレイアウトの崩れに関しては,M PLUSなどのWebフォントを使用してもらうことで解決できた.しかし,すべてのページを統合したファイルは大きすぎてGoogleスライド上ではPDF出力することができず,2019年度は担当者がMicrosoft PowerPointで出力することで対応した.2020年度は各セクションごとに統合するなどの措置を取ったほうが良いと考えている.

内容に関しては,一般の方に向けて頒布するため,読みやすく,また読んでいて楽しいものにすることを目標としていたが,こちらは概ね達成できていた.

提出に関しては,コラムは例年同様\thirdGrade{}以下の会員は原則全員提出としていた.提出期限を昨年よりも一週間ほど前倒しにして厳しい催促を行ったが,結果として非常に良い提出率となった.一部提出が遅れたものもいたが,担当者の編集に支障が出る遅延ではなく十分対応できた.

表紙デザインに関しては,担当者が有志に依頼するという形で制作を行った.2019年度担当者が2018年度担当者から2018年度は中のレイアウトもオマージュ元に合わせていたことを聞き損ねてしまったため,2019年度は文章のレイアウトは2018年度同様のものとなった.会誌全体のデザインとしては2018年度同様好評であった.

発注に関しては,2018年度から印刷業者を変更し担当者の負担の軽減を図った.印刷費用も2018年度の半分ほどになり,印刷冊数を増やすことができた.スケジュールは2018年度を参考に立てていたが11月は同人イベントが多く印刷業者の繁忙期となっていたため,学園祭までに印刷が間に合わなかった可能性もあった.対策としては入稿の予約を10月中旬までに入れておく,などが挙げられる.

2019年度初の試みとして,会誌のPDF版をWeb公開するということを行った.主に有志活動の際に外部のイベントなどで会員に利用され,より多くの人に会誌を読んでもらうことができた.

