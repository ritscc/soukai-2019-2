\subsection*{プロジェクト活動総括}
\writtenBy{\kensuiStaff}{中川}{拓海}

本局の方針は,会員のプロジェクト活動を円滑に進行する為,進捗確認や活動場所確保などのサポートを行うことである.

\subsubsection*{企画書の募集}
	
プロジェクト作成にあたり,企画書の作成を会員に呼び掛けた.
2019年度春学期の継続プロジェクトに加えて積極的な企画書の提出が行われた為,研究推進局が提出の催促を行う必要はなかった.
一回生がプロジェクトリーダーを務めるプロジェクト企画書も提出されていることから,一回生に対するプロジェクト活動についての周知は十分に行えていたと考えられる.

\subsubsection*{週報の回収・催促}

プロジェクトの進捗を管理する目的で週報の提出を各プロジェクトリーダーに義務付けた.
週報の回収にはGoogleフォームを利用し,週報の提出が為されていない班に対してはSlackおよび定例会議を通じて提出を促した.
2019年度秋学期の週報提出状況は著しく悪く,リマインドの強化を行ったが,平均で4週間分しか提出されなかった.
結果的に提出状況は改善しなかったものの,各班は毎週規定通りの活動を行えていた.
また,Slackを通じてプロジェクトリーダーから要求があった場合に活動部屋の確保を行った.活動部屋は毎週確保できていた.

\subsubsection*{会員のプロジェクト管理}

本局では,各会員がどのプロジェクトに所属しているかを把握し,プロジェクトが途中で終了した場合などに所属していた会員のプロジェクト異動などを管理している.
プロジェクト発足当初は後期入部者が複数名未配属のままであったが,局長および局員が対応して最終的には全会員がそれぞれ希望するプロジェクトに配属された.
また,プロジェクト間の会員の異動は発生しなかった.
2019年度春学期から継続で行われたプロジェクトに関して,DTM班は前リーダーの影響により活動自体が危ぶまれたものの,上回生会議にてプロジェクトの解散・継続が議論され,リーダーを変更することにより活動が継続されることとなった.
上記の理由より,DTM班は活動および週報の提出が停滞していた時期があったが,各会員のプロジェクト活動は滞りなく行われ,参加率は例年通りであった.

\subsubsection*{発表の機会の提供}

プロジェクト活動の成果発表をプロジェクト発表会を通じて行った.
2019年度秋学期は2019年度春学期での反省を踏まえて追い込み合宿とプロジェクト発表会の間に一定の期間を設けた結果,全ての報告書が提出され,全プロジェクトの発表が円滑に行えた.
問題点としては,発表会場の周知不足,制限時間のオーバー,報告書の修正箇所の指摘の有無,質疑応答がない,などが挙げられる.
これらに関しては本局での運営に問題があったため,2020年度からは局内での情報の共有を徹底する必要があると考えられる.