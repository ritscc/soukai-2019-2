\subsection*{全体総括}
\writtenBy{\kensuiChief}{服部}{瑠斗}
%\writtenBy{\kensuiStaff}{服部}{瑠斗}

今年度後期の研究推進局は以下の3点を目的として活動を行った.
\begin{itemize}
\item 平常活動の支援
\item 会員が興味関心のある活動ができる環境づくり
\item 発信力を養うための環境づくり
\end{itemize}

\subsubsection*{平常活動の支援}
平常活動の支援に関しては,プロジェクト活動の進捗管理やサポート,追い込み合宿,プロジェクト発表会の準備と進行を行った.
プロジェクト活動の進捗管理では,活動で生じた問題を週報で抽出し,上回生会議の議題に上げた.
今年度後期はDTM班の活動に大きな支障が発生したため,上回生会議の議題で取り扱った.
プロジェクト活動のサポートでは,活動に使用する部屋の予約手続きを行った.
追い込み合宿の準備では,宿泊の予約を行い,エポック立命21の宿泊部屋を確保した.
また,エポック立命21の交流室の確保を実施した.
しかし,交流室の解放予定時間と実際に開放された時間に差が生じしてしまった.
これは,研究推進局の局員が交流室の解放時間を失念していたことが原因として考えられる.
プロジェクト発表会では,部屋の確保などの準備や発表会の司会進行を行った.
しかし,プロジェクト発表会に際して, 発表者の作成した発表スライドが例年に比べて十分な内容を盛り込みすぎたため司会進行に多少の支障が発生してしまった.
これは,発表時間の周知を満足に実施していなかったことが原因だと考えられる.

\subsubsection*{会員が興味関心のある活動ができる環境づくり} 
会員が興味関心のある活動ができる環境づくりに関しては,目的を達成できなかったと考えられる.
これは,毎年開催されている勉強会を今年度後期に開催することが出来なかったことが原因だと考えられる.
開催出来なかった原因としては,研究推進局が勉強会の募集などの周知が不足していたことが挙げられる.

\subsubsection*{発信力を養うための環境づくり} 
発信力を養うための環境づくりに関しては,毎週の定例会議でライトニングトーク(以下,LT)を行った.
毎週の定例会議の時点で次週以降の担当者を通知したため,LTのための準備の時間を与えられたと考える.
今年度前期と比べると,LTを最後まで実施しない人は減少した.
これは,研究推進局が今年度前期の研究推進局の方針を踏まえてLTの催促を強く実施したことが原因だと考えられる.

また,LT意欲向上のため,LTアンケートを行い,入賞者には本会で購入する本の選択権を与えた.
今年度後期は,アンケートは実施し結果の公表までスムーズに実施することが出来た.
これは,研究推進局が定例会議などのスケジュールを把握して事前に行動していたことが原因だと考えられる.

