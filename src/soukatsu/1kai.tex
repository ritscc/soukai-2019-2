\subsection*{\firstGrade{}総括}

%\writtenBy{\firstGrade}{姓}{名}
%\writtenBy{\secondGrade}{姓}{名}
%\writtenBy{\thirdGrade}{姓}{名}
%\writtenBy{\fourthGrade}{姓}{名}

\writtenBy{\firstGrade}{黒柳}{裕大}
%\writtenBy{\secondGrade}{黒柳}{裕大}
%\writtenBy{\thirdGrade}{黒柳}{裕大}
%\writtenBy{\fourthGrade}{黒柳}{裕大}

%\writtenBy{\firstGrade}{黒柳}{裕大}
%\writtenBy{\secondGrade}{黒柳}{裕大}
%\writtenBy{\thirdGrade}{黒柳}{裕大}
%\writtenBy{\fourthGrade}{黒柳}{裕大}

2019年度の活動総括として現一回生の内情,プロジェクト活動,運営参加の3点を報告する.
2019年度に入学をした一回生は,サークル活動において積極性を持つ者が多かったため,新しい知識に頻繁に触れることができた.
また多くの人が回生コンパに出席したため,一回生内の親睦を深めることができたと言える.
プロジェクト活動においては2019年度春学期と2019年度秋学期,加え,参加率と達成度の4点から判断すると概ね満足な結果と言える.
しかし,学園祭の自主制作物が少なかったことから,自発的な活動が頻繁に行われていたとは言い難い.
運営参加に関しては,局配属は滞りなく進み,局会議の出席率も良かった.
しかし,局長決めに関しては消極的な方法で決めていた局があった.
学園祭の運営については,各々役割を果たしていた.

