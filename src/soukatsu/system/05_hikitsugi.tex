\subsection*{引き継ぎ総括}

\writtenBy{\systemStaff}{宇佐}{基史}

要点は以下の3点に分けられる.
 \begin{itemize}
\item RCC Wikiの読み合わせ
\item 停電対応
\item 現行システムの把握
 \end{itemize}
\subsubsection* {RCC Wikiの読み合わせ}
数度にわたりWikiの読み合わせを行った.また,読み合わせの参加が困難であった局員も各自で自主的に確認作業を行った.
しかし,新たな書き足しを含むWikiの更新作業は十分量行うことはできなかった.
\subsubsection* {停電対応}
停電対応に関しては本局のオブザーバである藤原が用意したマニュアルに沿って行われた.
しかし,対応作業当日に起きたバックアップが終了しないという事態へは対応ができなかったため,改善の余地が残されている.
\subsubsection* {現行システムの把握}
全ての現行システムを早急に下回生のみで運用できるように引き継ぎを完了させなくてはいけない.
2019年度は一部のシステムにおいて,全容を把握している\systemStaff{}が限定されてしまうという状況が生じていた.
その為,数度にわたり運用の上で上回生の判断を仰ぐ必要に迫られた.
