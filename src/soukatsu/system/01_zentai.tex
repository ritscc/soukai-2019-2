\subsection*{全体総括}

\writtenBy{\systemChief}{西見}{元希}
%\writtenBy{\systemStaff}{西見}{元希}

2019年度度第一回総会にて,全体方針として会内方針と局内方針の二つを立てた.会内方針は以下の3点であった.
\begin{itemize}
  \item 情報設備・システムの管理と維持
  \item 会内での快適な環境
  \item 本会採用サービスについての知識向上
\end{itemize}

情報設備・システムの管理と維持に関しては,大きな問題が発生することはなく十分な業務が行えていた.
ただし問題点として,春学期に移行したGitホスティングサービスであるGitlabに関して,一旦の運用は行えているが会員アカウントの管理体制がまだ整っていないというものと,会員に対してのMicrosoft Imagineサービスの提供を行うことができなかったことが挙げられる.
後者の詳細に関しては後のImagine総括にて述べる.

会内での快適な環境に関しては,数年前よりサークルルームを監視するためのアプリケーションとしてSlack上で動作していたWatcherをシステム管理局の提供サービスとすべく移行作業を行った.また,サークルルーム外で利用できるWi-FiアクセスポイントのSSIDの変更なども行い,会員がより快適に本会採用サービスを利用できるような環境を整えることができたと言える.

本会採用サービスについての知識向上に関しては,春学期に引き続き定例会議やサークルルームでの個別告知にて周知活動を行った.前述したWi-Fiアクセスポイントの利用方法の変更の際も会員に広く周知することができており,会内への情報共有は滞りなく行われていたと言える.

局内方針は以下の4点であった.
\begin{itemize}
  \item 局員がシステム管理を行うために必要な知識の向上
  \item 作業の効率化
  \item サーバ管理の徹底
  \item クライアントPCの最適化
\end{itemize}

局員がシステム管理を行うために必要な知識の向上に関しては,数回の局員勉強会を通しWikiの引き継ぎ文章の読み合わせを行った.
これにより知識は身につけることができたが運用にはまだ結びついていないとの声も局員から上がっており,引き継ぎを2020年度春学期も継続して行っていく必要がある.
作業の効率化に関しては,春学期同様業務を分散することができていた.
サーバ管理の徹底に関しては,大きな問題はなく,発生した問題も局会議の議題に上げ,情報を共有することができていた.
クライアントPCの最適化に関しては,従来設備の換装などは行わなかったが春学期までDTM設備として利用していたPCを有志がサービス提供を行うことのできるPCとして再利用した.
