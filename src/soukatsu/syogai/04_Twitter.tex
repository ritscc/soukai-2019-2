\subsection*{Twitter総括}

\writtenBy{\syogaiChief}{堀越}{俊行}
%\writtenBy{\syogaiStaff}{堀越}{俊行}

会公式Twitterの運用については,主に以下のことに関して投稿した.
\begin{itemize}
\item 定例会議でのライトニングトーク(以下,LT)
\item ハッカソン等の本会に関するイベントの告知及び実況
\item 会員の製作物紹介
\end{itemize}

\subsubsection*{定例会議でのLTについて}
LTに関しては,発表終了後はすぐに文章を作成し,発表者本人によるレビュー後即座に投稿することは概ねできていた.
また被写体を載せてよいかの確認は必ず行ったため,肖像権に配慮した投稿ができたと考えられる.
しかし,事前に担当者を割り振ることができておらず,その場での対応が多く見られた.

\subsubsection*{ハッカソン等の本会に関するイベントの告知及び実況}
本会の後期のイベントについては学園祭があった.
これについては開催場所の事前投稿や当日の実況ができていた.

ハッカソンや勉強会に関しては,実況や発表者に許可が取れた内容を投稿をする事ができた.

\subsubsection*{会員の製作物紹介}
会員の製作物紹介に関しては,作者がLTの中で紹介することが多く,
またLT以外の製作物発表がなかったため,投稿は1度も行わなかった.

\subsubsection*{各人によるTwitterの運用}
各人の操作ミスや確認不足により、誤ったいいねやリツイートがあった.
また,DM対応なども各人の確認時期が偏っていたため,数度企業などからのDM対応の遅れが見られた.
