\subsection*{会公式Twitter総括}

\writtenBy{\syogaiChief}{堀越}{俊行}
%\writtenBy{\syogaiStaff}{堀越}{俊行}

会公式Twitterの運用については,主に以下のことに関して投稿した.
\begin{itemize}
\item 定例会議でのLT
\item ハッカソンなどの本会に関するイベントの告知及び実況
\item 会員の製作物紹介
\end{itemize}

\subsubsection*{定例会議でのLTについて}
定例会議のLTについては,発表後すぐに文章を作成し,発表者本人によるレビューの後即座に投稿することを目標としていたが,こちらは概ね達成できた.
また被写体を載せてよいかの確認は必ず行ったため,肖像権に配慮した投稿ができたと考えられる.
しかし,事前に担当者を割り振ることができておらず,その場での対応が多く見られた.
これによる投稿漏れなどの問題は見られなかった.

\subsubsection*{ハッカソンなどの本会に関するイベントの告知及び実況}
本会の2019年度秋学期のイベントについては学園祭があった.
これについては開催場所の事前投稿や当日の実況が行えていた.

ハッカソンや勉強会に関しては,実況や発表者に許可が取れた内容を投稿することができた.

\subsubsection*{会員の製作物紹介}
会員の製作物紹介に関しては,作者がLTの中で紹介することが多く,
またLT以外の製作物発表がなかったため,投稿は1度も行わなかった.

\subsubsection*{局員によるTwitterの運用}
局員の操作ミスや確認不足により,誤ったいいねやリツイートがあった.
また,ダイレクトメッセージ対応なども局員の確認時期が偏っていたため,数度企業などからのダイレクトメッセージ対応の遅れが見られた.
2019年度秋学期のダイレクトメッセージ対応は多くが宣伝目的のものであったため,特に大きな問題はなかった.
