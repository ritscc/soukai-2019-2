\subsection*{\thirdGrade{}総括}

\writtenBy{\thirdGrade}{稲垣}{和真}

2019年度秋学期の\thirdGrade{}方針は,次の3点であった.
\begin {itemize}
	\item 下回生の運営サポート
	\item 適度に活動しやすい雰囲気づくり
    \item 下回生の模範となれる行動
\end {itemize}

\subsubsection*{下回生の運営サポート}

1点目の下回生の運営サポートについては,
\secondGrade{}が本会の運営の中心となることから,
企画運営,通常時の運営それぞれが円滑に進むよう補助することであった.
総じて,\thirdGrade{}の補助は,\secondGrade{}が運営の主体であるという認識のもと,
運営に過干渉にならないように努める方針であった.

\secondGrade{}が主体的に活動を行っていたことから,\thirdGrade{}からの積極的な支援は必要にならなかった.
また,2019年度第2回総会の引き継ぎに関しては,以下の2点を行った.

\begin {itemize}
	\item 担当の割り振り
	\item 考慮すべき点の伝達
\end {itemize}

これらを考慮した結果,概ね下回生の運営サポートは達成できたと言える.
しかし,運営のためのマニュアルの整備に関しては未完了であることから,一刻も早く整備をすることが望ましい.

\subsubsection*{適度に活動しやすい雰囲づくり}
\thirdGrade{}は,原則あまり干渉しないように努めた.
しかし,メリハリを意識してもらうための注意喚起は行うようにした.
これにより,局会議などの会内の重大な事柄の進行が円滑になるように努めた.
また,本会の活動外にて食事などの交流も行うことで親睦も深められ,
居心地の良い空間を作ることが出来た.
以上より,本目的は達成できたと考えている.

\subsubsection*{下回生の模範となれる行動}
遅刻欠席などの連絡については,\thirdGrade{}は到着予想時刻などの連絡をしており,比較的下回生の模範となるような丁寧な連絡ができていたと考えられる.
LTにおいては,自由な話題性を持って積極的に発表を行うことで,LTへの敷居を下げることが出来たと考えている.
実際に\firstGrade{}の参加率も良く,会内でのLTの活性化が行えたと考え,
自由な情報発信の楽しさを伝えられた.
以上より,本目的は達成できたと考えている.
